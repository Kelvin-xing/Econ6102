% This is a simple template for a LaTeX document using the "article" class.
% See "book", "report", "letter" for other types of document.

\documentclass[10pt]{article} % use larger type; default would be 10pt

\usepackage[utf8]{inputenc} % set input encoding (not needed with XeLaTeX)

%%% Examples of Article customizations
% These packages are optional, depending whether you want the features they provide.
% See the LaTeX Companion or other references for full information.

%%% PAGE DIMENSIONS
\usepackage{geometry} % to change the page dimensions
\geometry{a4paper} % or letterpaper (US) or a5paper or....
\usepackage{setspace}
\usepackage{parskip}
\parskip = 0.3 \baselineskip %\advance\parskip by 0pt plus 2pt% to change between paragraphs space
% \geometry{margin=2in} % for example, change the margins to 2 inches all round
% \geometry{landscape} % set up the page for landscape
%   read geometry.pdf for detailed page layout information

% \usepackage{gravarphicx} % support the \includegravarphics command and options
% \usepackage[parfill]{parskip} % Activate to begin paragraphs with an empty line rather than an indent

%%% PACKAGES
\usepackage{{booktabs}} % for much better looking tables
\usepackage{array} % for better arrays (eg matrices) in maths
\usepackage{paralist} % very flexible & customisable lists (eg. enumerate/itemize, etc.)
\usepackage{verbatim} % adds environment for commenting out blocks of text & for better verbatim
\usepackage{subfig} % make it possible to include more than one captioned figure/table in a single float
% These packages are all incorporated in the memoir class to one degree or another...
\usepackage[fleqn]{amsmath}
\usepackage{amssymb}
\usepackage{enumitem}
\usepackage{amsthm}
\usepackage{graphicx}
\usepackage{filecontents}
\usepackage{natbib}
\usepackage{blindtext}
\usepackage{titlesec}
\usepackage[table,xcdraw]{xcolor}


%%% HEADERS & FOOTERS
\usepackage{fancyhdr} % This should be set AFTER setting up the page geometry
\pagestyle{plain} % options: empty , plain , fancy
\renewcommand{\headrulewidth}{0pt} % customise the layout...
\lhead{}\chead{}\rhead{}
\lfoot{}\cfoot{\thepage}\rfoot{}

%%% SECTION TITLE APPEARANCE
\usepackage{sectsty}
\allsectionsfont{\rmfamily\bfseries\upshape} % (See the fntguide.pdf for font help)
% (This matches ConTeXt defaults)

%%% ToC (table of contents) APPEARANCE
\usepackage[nottoc,notlof,notlot]{tocbibind} % Put the bibliography in the ToC
\usepackage[titles,subfigure]{tocloft} % Alter the style of the Table of Contents
\renewcommand{\cftsecfont}{\rmfamily\mdseries\upshape}
\renewcommand{\cftsecpagefont}{\rmfamily\mdseries\upshape} % No bold!

\usepackage[colorlinks,citecolor=black,urlcolor=black,bookmarks=false,hypertexnames=true]{hyperref} 

%%% END Article customizations



%%% The "real" document content comes below...

\title{MECON6102 Problem Set 2}
\author{Xing Mingjie}
\date{\today} % Activate to display a given date or no date (if empty),
         % otherwise the current date is printed 

\begin{document}
\maketitle

% \tableofcontents

\begin{abstract}
    This report studies the relationship between monthly stock returns and a set of factors and construct an investment strategy based on various factor models. The report first uses a naive factor regression to estimate the factor loadings of the stock returns. Then, the report uses the Fama-MacBeth regression to estimate the factor risk premia. Finally, the report uses the LASSO regression to select the factors. The report also constructs a mean-variance portfolio based on the factor risk premia.
\end{abstract}

\section{Data}
    \subsection{Description}
    \begin{table}
\caption{Data Description}
\label{tab:data_description}
\begin{tabular}{lrrrrr}
\toprule
 & count & mean & std & min & max \\
\midrule
\textbf{default\_label} & 13982.00 & 0.02 & 0.15 & 0.00 & 1.00 \\
\textbf{age} & 13982.00 & 41.66 & 14.56 & 17.00 & 66.00 \\
\textbf{gender} & 13982.00 & 0.46 & 0.50 & 0.00 & 1.00 \\
\textbf{edu} & 13982.00 & 1.69 & 1.10 & 0.00 & 4.00 \\
\textbf{housing} & 13982.00 & 0.63 & 0.48 & 0.00 & 1.00 \\
\textbf{income} & 13982.00 & 7426.48 & 6226.68 & 650.42 & 37515.37 \\
\textbf{job\_occupation} & 13982.00 & 0.34 & 0.56 & 0.00 & 2.00 \\
\textbf{past\_bad\_credit} & 13982.00 & 0.96 & 0.19 & 0.00 & 1.00 \\
\textbf{married} & 13982.00 & 0.53 & 0.50 & 0.00 & 1.00 \\
\bottomrule
\end{tabular}
\end{table}


    Table \ref{tab:data_description} shows the summary statistics of the data. The data set contains 13982 observations and 9 variables. The dependent variable is the default label, which is a binary variable indicating whether the individual defaults. 

    \begin{figure}
        \centering
        \includegraphics[width=0.8\textwidth]{"../fig/variable_heatmap.png"}
        \caption{Heat map of the correlation matrix of the variables}
        \label{fig:variable_heatmap}
    \end{figure}
    Figure \ref{fig:variable_heatmap} shows the heat map of the correlation matrix of the features and target. Most of the features are arguably uncorrelated. There is a high correlation between housing and age at 0.55. The correlation between income and education level is 0.51, which captures the wage premium of education.
    \subsection{Data Preprocessing}
    \begin{figure}
        \centering
        \includegraphics[width=0.8\textwidth]{"../fig/income_distribution.png"}
        \caption{Income Distribution}
        \label{fig:income_skew}
    \end{figure}
    Figure \ref{fig:income_skew} shows the distribution and the skewness of feature \texttt{income}. The distribution is right-skewed. The report uses the log transformation to reduce the skewness of the feature for better performance in models.
    \begin{figure}
        \centering
        \includegraphics[width=0.8\textwidth]{"../fig/income_distribution_log_transform.png"}
        \caption{Income Distribution After Log Transformation}
        \label{fig:income_log}
    \end{figure}
    Figure \ref{fig:income_log} shows the distribution of the feature \texttt{income} after the log transformation. The distribution is more symmetric after the transformation.

\section{Model Comparison}
    \subsection{Simple Logistic Model}


%% put here a table of the model comparison on AUC
\section{Conclusion}

\newpage
\footnotesize
\bibliographystyle{apalike}
\bibliography{ref}

\end{document}